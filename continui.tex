\documentclass[a4paper,11pt]{report}
%\usepackage{fontspec}
%\setmainfont{Exo}

%stile
\usepackage{fancyhdr}
\usepackage[a4paper, top=2.5cm, bottom=2cm, left=1.5cm, right=1.5cm]{geometry}
\usepackage[italian]{babel}
\usepackage[T1]{fontenc}
\usepackage[utf8]{inputenc}

%Immagini, grafici
\usepackage{wrapfig}
\usepackage{graphicx}
\usepackage{subfigure}
\usepackage{caption}

\usepackage{lipsum}
\usepackage{verbatim}
\usepackage{siunitx}
\usepackage{hyperref}

\usepackage{amsmath,amssymb,amsthm} % matematica
\usepackage{booktabs}
\usepackage{array} %colonne di larghezza prefissata per tabular
\newcolumntype{L}[1]{>{\raggedright\arraybackslash}p{#1}}
\newcolumntype{C}[1]{>{\centering\arraybackslash}p{#1}}
\newcolumntype{R}[1]{>{\raggedleft\arraybackslash}p{#1}}

\title{Note del corso di Fisica dei Continui}
\author{trascritte da \\Luca Colombo Gomez}
\date{AA 2019/2020}

\newcommand{\eacc}{È }
\newcommand{\defeq}{\overset{\cdot}{=}}


\newcommand{\R}{\mathbb{R}}
\newcommand{\Rn}{\mathbb{R}^n}
\newcommand{\fourier}{\emph{F}}
\newcommand{\x}{\bar{x}}
\newcommand{\xt}{\bar{x}(t)}
\newcommand{\vel}{\bar{v}}
\newcommand{\vt}{\bar{v}(t)}
\newcommand{\xp}{\bar{x}'}
\newcommand{\y}{\bar{y}}
\newcommand{\yp}{\bar{y}'}
\newcommand{\kk}{\bar{k}}
\newcommand{\kp}{\bar{k}'}
\newcommand{\z}{\bar{z}}
\newcommand{\n}{\bar{n}}

\begin{document}
	\titlepage
	\maketitle
	\tableofcontents
	
\chapter*{Memorie della termodinamica}
		Il primo principio della termodinamica, visto come legge di bilancio energetico, parte dall'energia interna U di un sistema, la cui equazione dU è in generale scrivibile come
		$$dU = TdS - pdV + \mu dN $$
		(con $\mu$ potenziale chimico e N numero molare; se ci sono più componenti scriviamo $\sum_i \mu_i dN_i $, ma la sostanza è la stessa)
		e allora per un sistema termostatato isolato che non scambia altro che calore (dV =0, dN = 0)
		$$
		\delta Q = dU = TdS \rightarrow dS = \dfrac{\delta Q}{T}
		$$
		sia la trasformazione reversibile o meno.
		
		\eacc utile osservare di U, che essa è funzione di S,V,N, tutte grandezze estensive, ed è esa stessa estensiva, vale a dire 
		\begin{equation}
			U(\lambda S, \lambda V, \lambda N) = \lambda \cdot U(S,V,N) \qquad \forall \lambda \in \Re^+
		\end{equation}
		ovvero U è una funzone omogenea di grado 1. Per le funzioni omogenee di grado n si ha $f(\lambda x) = \lambda ^n f(x)$, o anche, data $f(x_1,\dots, n_m)$
		\begin{equation}
			\sum_{i=1}^{m}x_i \dfrac{\partial}{\partial x_i} f(x_1, \dots, x_m) = n \cdot f(x_1, \dots, x_m)
		\end{equation}
		che, applicato a U(S,V,N), da
		\begin{equation}
			U(S,V,N) = \dfrac{\partial U}{\partial S} S + \dfrac{\partial U}{\partial V} V + \dfrac{\partial U}{\partial N} N
		\end{equation}
		Da $dU = TdS -pdV + \mu dN$ si ottengono le \underline{equazioni di stato}
		\begin{equation}
			T=\dfrac{\partial U}{\partial S}; \qquad p = - \dfrac{\partial U}{\partial V}; \qquad \mu = \dfrac{\partial U}{\partial N}
		\end{equation}
		e per confronto con la relazione di Eulero si ha
		\begin{equation}
			U(S,V,N)=TS-pV+\mu N
		\end{equation}
	\section{Potenziale di Helmholtz F ( o energia libera di Helmoltz)}
			F da una misura del lavoro estraibile da un sistema chiuso a T,V costanti (vediamo sotto).
			Si ottiene come trasformazione di Legendre di U rispetto a S:
			\begin{equation}
				F \overset{\cdot}{=} U- \dfrac{\partial U}{\partial S} S = U-TS
			\end{equation}
			e si ha, da $U(S,V,N)$, una funzione $F = F(\frac{\partial U}{\partial S} \rightarrow T, V,N)$
			
			Inoltre 
			\begin{subequations}
				\begin{equation}
				F = U - TS = TS - pV + \mu N -TS = -pV + \mu N
				\end{equation}
				\begin{equation}
				dF = dU - TdS - SdT = TdS - pdV + \mu dN - TdS - SdT = -SdT - pdV + \mu dN
				\end{equation}
			\end{subequations}
			Possiamo dunque dimostrare l'affermazione iniziale. Preso un sistema S formato da costituenti in contatto termico con un termostato R a temperatura $T^R$ ( reservoir infinito di calore), il lavoro massimo estraibile dal sistema è
			\begin{equation}
				dL^{max} = -dF^S
			\end{equation}
			Infatti 
			$$
			dL = -dU^S - dU^R \underset{\text{R termostato}}{\underbrace{=}} -dU^S - T^RdS^R
			$$
			Poiché il sistema non scambia calore con l'esterno, l'entropia non decresce:
			$$
			dS^S + dS^R \geq 0 \implies -dS^R \leq dS^S
			$$
			$$
			\implies dL = -dU^S - T^R dS^R \leq -dU^S + T^R dS^S = -d(U^S - T^R S^S) \underset{T^R = T^S \textrm{(contatto termico)}}{\underbrace{=}} -d(U^S - T^S S^S)
			$$
			cioè $ dL\leq -dF^S$, il cui massimo appunto è $dL^{max}=-dF^S$ (caso di trasformazione reversibile)\\
			
			Osservazioni: 
			\begin{itemize}
				\item ne consegue che lo stato di equilibrio del sistema termostatato è quello che minimizza l'energia libera di Helmoltz; (in equivalente è minima U, $dU = dU^S + d^R = 0 \implies dF=0, F\textrm{F stazionaria}$)
				\item F (come tutti i potenziali termodinamici) è definito all'equilibrio - altrimenti si deve supporre un insieme di sottosistemi in equilibrio e isotermi;
				\item  esempio di lavoro - energia libera è il lavoro molare di estrazione di un metallo, che è pari alla variazione di F in una mole di $e^-$ nel passare dal metallo all'esterno
			\end{itemize}			
		
	
	\section{Entalpia H}
			H da una misura del lavoro estraibile da un sistema chiuso a p costante.\\
			Si ottiene come trasformazione di Legendre di U rispetto a V:
			\begin{equation}
			H \overset{\cdot}{=} U- \dfrac{\partial U}{\partial V} V = U + pV
			\end{equation}
			
			e si ha, da $U(S,V,N)$, una funzione $H = H(,S\frac{\partial U}{\partial V} \rightarrow p,N)$\\
			
			Inoltre 
			\begin{equation}
				\begin{aligned}
				H = U + pV = TS -pV +\mu N + pV = TS + \mu N \\
			 	dH = dU +pdV + Vdp = TdS - pdV + \mu dN + pdV + Vdp = TdS + Vdp + \mu dN
				\end{aligned}
			\end{equation}
			Se il sistema è chiuso (non scambia massa) e mantenuto a p costante, $dH = TdS = \delta Q_{rev}$ il calore assorbito in una trasformazione reversibile è pari alla variazione di H.\\
			
			Nota: è questo il caso delle trasformazioni gatte a T costante; infatti p è costante, i potenziali chimici sono uguali\footnote{ce lo dice Josiah Willard Gibbs alla pagina seguente}, e il numero di moli (la massa) nel complesso è conservato$(dN^{(1)} + dN^{(2)} = 0) \implies Vdp + \sum_i \mu^{(i)} dN^{(i)} = 0$, da cui vediamo che il calore latente di trasformazione è la differenza di entalpia per unità di massa tra le fasi.\\
			Similmente al caso dell'energia libera, si può dimostrare che preso un sistema chiuso S di costituenti mantenuti a p costante da un reservoir di pressione a $p=p^R$, il lavoro massimo estraibile da S è 
			$$
			dL^{max} = -dH^S
			$$
			e lo stato di equilibrio di un sistema mantenuto a p costante di un reservoir di pressione è quello che minimizza l'entalpia.
			
			Nota: Landau chiama $w \overset{\cdot}{=} H/M$ entalpia per unità di massa $(w = \varepsilon + pv = \varepsilon + p/\rho);\\
			v \defeq V/M = 1/\rho$ volume specifico
		
	\section{Potenziale di Gibbs G ( o energia libera di Gibbs)}
		G da una misura del lavoro estraibile da un sistema chiuso a T,p costanti. Si ottiene come trasformazione di Legendre di U rispetto a S e V:
		$$
		G \defeq U - \dfrac{\partial U}{\partial S}S - \dfrac{\partial U}{\partial V}V = U -TS + pV
		$$
		
		e si ha, da $U(S,V,N)$, una funzione $H = H(\dfrac{\partial U}{\partial S}\rightarrow T,\frac{\partial U}{\partial V} \rightarrow p,N)$\\ 
		
		Inoltre 
		\begin{equation}
			G = U -TS + pV = TS - pV + \mu N - TS + pV = \mu N
		\end{equation}
		($\mu = G/N $ poteniale di Gibbs molare coincide con potenziale chimico)
		\begin{equation}
			dG = dU -TdS - SdT + pdV + Vdp = TdS - pdV + \mu dN - TdS - SdT + pdV + Vdp = -SdT + Vdp + \mu dN
		\end{equation}
		
		Si può dimostrare che preso un sistema chiuso S i cui costituenti sono mantenuti a T e p costanti da reservoir di temperatura (termostato) a $T^R$ e di pressione a $p^R$, il lavoro massimo estraibile da S è 
		$$
		dL^{max} \leq -dG^S
		$$
		e lo stato di equilibrio di un sistema a T e p costanti grazie a reservoir ideali di T,p è quello che minimizza il potenziale di Gibbs.
		
		Nota: dato un sistema a due fasi in equilibrio, se una tra p e T è fissata, lo è anche l'altra, e $\mu^{(1)} = \mu^{(2)}$.\\
		Ciò perchè 
		$$
		G = \mu^{(1)} dN^{(1)} + \mu^{(2)} dN^{(2)}
		$$
		e
		$$
		dG = \mu^{(1)} dN^{(1)} + \mu^{(2)} dN^{(2)}
		$$
		ma la massa complessiva non varia $\implies N^{(1)} + N^{(2)} = cost \iff dN^{(1)} + dN^{(2)} = 0$
		$$
		\implies dG = \mu^{(1)} dN^{(1)} - \mu^{(2)} dN^{(1)} = (\mu^{(1)} - \mu^{(2)})dN^{(1)}
		$$
		$$
		\textrm{Equilibrio} \iff dG =0 \implies \mu^{(1)}(p,T) = \mu^{(2)}(p,T)
		$$
		Questa, essendo i $\mu^{(i)}$ rappresentativi di fasi diverse e di funzioni diverse, rappresenta un'espressione implicita della relazione tra p e T; se p fissata $\implies$ determinata anche T.
		
		Nota: Landau chiama $\phi \defeq G/M$ energia libera di Gibbs per unità di massa\\
		
		Poiché 
		$$
		G= \sum_i \mu^{(i)} N^{(i)} \implies dG = \sum_i \mu^{(i)} dN^{(i)} + \sum_i N^{(i)} d\mu^{(i)}
		$$
		ma anche da 
		$$
		G = U - TS + pV \implies dG = -SdT + Vdp + \sum_i \mu^{(i)} dN^{(i)}
		$$
		Si ottiene l'equazione di Gibbs - Duhem
		\begin{equation}
			\sum_i N^{(i)} d\mu^{(i)} = -SdT + Vdp
		\end{equation}
		Si può anche derivare l'equazione di Clausius - Clapeyron, che descrive la pendenza della curva di equilibrio $dp/dT$ tra due fasi di una sostanza.
		
		Le due fasi in equilibrio, 1 e 2, hanno lo stesso potenziale di Gibbs:
		\begin{multline}
			G_1(T,p) = G_2 (T,p) \implies dG_1(p,T)= dG_2(p,T) \\
			\implies \dfrac{\partial G_1}{\partial p} dp + \dfrac{\partial G_1}{\partial T}dT = \dfrac{\partial G_2}{\partial p}dp + \dfrac{\partial G_2}{\partial T}dT
		\end{multline}

		Da $dG = -SdT + Vdp + \sum_i \mu^{(i)} dN^{(i)}$:
		$$
		\left\{\begin{matrix}
		\dfrac{\partial G}{\partial T} = -S \\ 
		\dfrac{\partial G}{\partial p} = V 
		\end{matrix}\right.
		$$
		\begin{equation}
			V_1 dp -S_1 dT = V_2 dp - S_2 dT \implies \dfrac{dp}{dT} = \dfrac{S_1 - S_2}{V_1 - V_2}
		\end{equation}
		
		A T costante
		
		\begin{equation}
			S_2 - S_1 = \int_{1}^{2}\dfrac{\delta Q}{T} = \dfrac{1}{T} \int_{1}^{2} \delta Q = \dfrac{\lambda M}{T}
		\end{equation}
		
		Con $\lambda$ calore latente di trasformazione per unità ti massa, M=massa. Detto $v \defeq V/M$ volume specifico
		\begin{equation}
			\dfrac{dp}{dT} = \dfrac{\lambda}{T(V_2 - V_1)}
		\end{equation}
	
	\section{Relazioni di Maxwell}
		Lavoreremo tipicamente in sistemi senza scambio di massa, ignorando quindi il termine $\sum_i \mu_i dN_i$
		\begin{itemize}
		\item Dal primo principio della termodinamica
		$$
		U=U(S,V)
		$$
		$$
		dU = TdS - pdV \implies dU =\underset{T}{ \underbrace{\left.\dfrac{\partial U}{\partial S}\right|_{V=cost}}}dS +  \underset{-p}{\underbrace{\left.\dfrac{\partial U}{\partial V}\right|_{S=cost}}}dV
		$$
		$$
		T= \left.\dfrac{\partial U}{\partial S}\right|_{V=cost}; \qquad p= -  \left.\dfrac{\partial U}{\partial V}\right|_{S=cost}
		$$
		Per il teorema di Schwarz, vale $\dfrac{\partial^2 U}{\partial S\partial T} = \dfrac{\partial^2 U}{\partial T \partial S} $ quindi
		$$
		\left.\dfrac{\partial T}{\partial V}\right|_{S=cost} = \dfrac{\partial}{\partial V}\left( \left.\dfrac{\partial U}{\partial S}\right|_V \right)_S = 
		\dfrac{\partial}{\partial S}\left( \left.\dfrac{\partial U}{\partial V} \right|_S \right)_V = - \left.\dfrac{\partial p}{\partial S}\right|_{V=cost}
		$$
		
		\begin{equation}
			\left.\dfrac{\partial T}{\partial V}\right|_{S=cost} = - \left.\dfrac{\partial p}{\partial S}\right|_{V=cost}
		\end{equation}
		
		\item Dal differenziale dell'entalpia
		$$
		H=H(S,p)
		$$
		$$
		dH = TdS + Vdp \implies dH =\underset{T}{ \underbrace{\left.\dfrac{\partial H}{\partial S}\right|_{p=cost}}}dS +  \underset{V}{\underbrace{\left.\dfrac{\partial H}{\partial p}\right|_{S=cost}}}dp
		$$
		$$
		T= \left.\dfrac{\partial H}{\partial S}\right|_{p=cost}; \qquad V= \left.\dfrac{\partial H}{\partial p}\right|_{S=cost}
		$$
		Per il teorema di Schwarz, uguagliando le due derivate parziali miste

		
		\begin{equation}
		\left.\dfrac{\partial T}{\partial p}\right|_{S=cost} = \left.\dfrac{\partial V}{\partial S}\right|_{p=cost}
		\end{equation}
		
		\item Dal differenziale del potenziale di Helmholtz
		
		$$
		F=F(T,V)
		$$
		$$
		dF = - SdT - pdV \implies dF =\underset{-S}{ \underbrace{\left.\dfrac{\partial F}{\partial T}\right|_{V=cost}}}dT +  \underset{-p}{\underbrace{\left.\dfrac{\partial F}{\partial V}\right|_{T=cost}}}dV
		$$
		$$
		S = -\left.\dfrac{\partial F}{\partial T}\right|_{V=cost}; 
		\qquad
		p = - \left.\dfrac{\partial F}{\partial V}\right|_{T=cost}
		$$
		Per il teorema di Schwarz, uguagliando le due derivate parziali miste
		
		\begin{equation}
		\left.\dfrac{\partial S}{\partial V}\right|_{T=cost} = \left.\dfrac{\partial p}{\partial T}\right|_{V=cost}
		\end{equation}
		
		\item Dal differenziale del potenziale di Gibbs
		
		$$
		G=G(T,p)
		$$
		$$
		dG = - SdT + Vdp \implies dG =\underset{-S}{ \underbrace{\left.\dfrac{\partial G}{\partial T}\right|_{p=cost}}}dT +  \underset{V}{\underbrace{\left.\dfrac{\partial G}{\partial p}\right|_{T=cost}}}dp
		$$
		$$
		S= -\left.\dfrac{\partial G}{\partial T}\right|_{p=cost}; 
		\qquad
		V= \left.\dfrac{\partial G}{\partial p}\right|_{S=cost}
		$$
		Per il teorema di Schwarz, uguagliando le due derivate parziali miste
		
		\begin{equation}
		\left.\dfrac{\partial S}{\partial p}\right|_{T=cost} = - \left.\dfrac{\partial V}{\partial T}\right|_{p=cost}
		\end{equation}
		
		\end{itemize}
	
	\section{Calori specifici}
		la capacità termica C di una porzione di materia M, è il rapporto tra calore $\delta Q$ assorbito e aumento di temperatura $dT$. La capacità termica per unità di massa è detta \underline{calore specifico} $C \defeq \delta Q/ dT$
		$$
		c \defeq \dfrac{1}{M}\dfrac{\delta Q}{d T}; \quad \textrm{oppure per mole} \quad C \defeq \dfrac{1}{N}\dfrac{\delta Q}{d T}
		$$
		Questo valore dipende dal tipo di processo in cui avviene lo scambio di $\delta Q$, ovvero a pressione o volume costante
		
		$$
		c_p = \left.\dfrac{1}{M}\dfrac{\delta Q}{d T}\right|_{p=cost} \quad \left( C_p = \left.\dfrac{1}{N}\dfrac{\delta Q}{d T}\right|_{p} \right); \qquad c_v=  \left.\dfrac{1}{M}\dfrac{\delta Q}{d T}\right|_{V=cost} \quad  \left( C_V = \left.\dfrac{1}{N}\dfrac{\delta Q}{d T}\right|_{V} \right)
		$$
		
		\begin{itemize}
			\item Dal 1 principio della termodinamica $dU = \delta Q - pd$; per una trasformazione quasi-statica a V costante $dU = \delta Q$
			$$
			\implies c_V =  \left.\dfrac{1}{M}\dfrac{\partial U}{\partial T}\right|_{V} =  \left.\dfrac{1}{M}\dfrac{\partial \varepsilon}{\partial T}\right|_{V}
			$$
			avendo $\varepsilon \defeq U/M$ energia interna per unità di massa
			
			\item lavorando a p costante sfruttiamo $dH = TdS  + Vdp$, e per trasformazione quasi-statica, reversibile, $dH = TdS = \delta Q$
			$$
			\implies c_p =  \left.\dfrac{1}{M}\dfrac{\delta Q }{\partial T}\right|_{p} =
			\left.\dfrac{T}{M}\dfrac{\partial S}{\partial T}\right|_{V} = 
			\left. T \dfrac{\partial s}{\partial T}\right|_{p}
			$$
			con $s\defeq S/M$
			
			\item Per gas perfetti, $pV = NRT$; $ H= U+pV = U + NRT$
			derivando rispetto a T
			$$
			\dfrac{\partial H}{\partial T} = \dfrac{\partial U}{\partial T} + NR
			$$
			$$
			\implies \dfrac{1}{N} \dfrac{\partial H}{\partial T} = \dfrac{1}{N}\dfrac{\partial U}{\partial T} + R
			$$
			$$
			C_p = C_V + R
			$$
		\end{itemize}		
	
	\section{Processi adiabatici}
		Una trasformazione è adiabatica quando avviene senza scambio di calore.\\
		Poiché $\delta Q =0 \implies dU + pdV =0$
		$$
		NC_VdT + \dfrac{NRT}{V}dV = 0
		$$
		dividendo per $NC_VT$
		$$
		\implies \dfrac{dT}{T} + \dfrac{R}{C_V} \dfrac{dV}{V}=0 \quad 
		\implies d\left(\log T\right) +  \dfrac{R}{C_V} d(\log V) = 0
		$$
		$$
		\implies d\left(\log T + \log V^{ \frac{R}{C_V}}\right) = 0
		$$
		$$
		\implies \log(TV^{R/C_V}) = \mathrm{costante} 
		\quad \implies \quad TV^{R/C_V} = \textrm{costante}
		$$
		Definito $\gamma \defeq \dfrac{C_P}{C_V}$ si ha
		$$
		\gamma = \dfrac{C_V + R}{C_V} = 1 + \dfrac{R}{C_V} \implies \dfrac{R}{C_V} = \gamma -1
		$$
		\begin{equation}
			TV^{\gamma -1} = \mathrm{costante}
		\end{equation}
		che è l'equazione della trasformazione adiabatica.
		
		Usando $pV = NRT$ si può scrivere equivalentemente
		$$
		T \propto pV \implies pV^{C_p/C_V}
		$$
		\begin{equation}
			pV^{\gamma} = \mathrm{costante}
		\end{equation}
		
		$$
		V\propto Tp^{-1} \implies T\left(\dfrac{T}{p}\right)^{\gamma-1} = T^{\gamma}p^{1-\gamma} = \textrm{costante} \implies Tp^{\frac{1-\gamma}{\gamma}}
		$$
		\begin{equation}
		Tp^{\frac{1}{\gamma -1}} = \mathrm{costante}
		\end{equation}
		
	
	

\chapter{Generalità sui continui}
		Particella o masserella fluida: porzione di mezzo continuo di dimensione lineare infinitesima.\\
		Continui: mezzo materiale in cui, una volta stabilita la massima lunghezza d\footnote{Diametro = distanza massima tra due punti del volume} che si possa trattare come infinitesima, presa una regione di diametro d, il numero di costituenti microscopici in essa contenuti è grande abbastanza da risultare statisticamente significativo.\\
		
		In un solido o liquidi, possiamo stimare che le distanza intermolecolari/interatomiche siano dell'ordine di $1-\SI{10}{\angstrom}$, cioè $0.1-1\,mm$; in un cubo di lato $1\,\mu m$ ci saranno perciò $>10^9$ costituenti elementari. 
		$1\,\mu m$ è perciò una lunghezza
		\begin{itemize}
			\item infinitesima se studiamo un fluido macroscopico, o al più oggetti nel fluido come granuli ($\sim 10-100\,\mu m$); OK trattazione come continuo.
			\item NON infinitesima per un nanotecnologo, che ha a che gare con scale $\leq 1\,\mu m$; NON va bene l'ipotesi di continuo.
		\end{itemize}
		
		Per un gaso come l'aria, una mole, cioè $\sim 10^{23}$ particelle, occupa $\approx 22$ litri in condizioni normali (1\,atm - $\SI{0}{\celsius}$), cioè un cubo di $\sim 0.28\,m$ di spigolo $\sim 10^5\,\mu m$; nel cubo di $\sim 1\mu m^3$ avremo $10^{23}/10^{15} = 10^8$ particelle, ancora un numero statisticamente sufficiente;
		ma nella parte più alta dell'atmosfera terrestre ($>100\,km$), con pressioni di un ordine di grandezza inferiori, la statistica non è più plausibile e non siamo in regime di continuo.
		
		Come studiare un fluido? Due punti di vista:
		\begin{itemize}
			\item Euleriano: osservo la variazione temporale delle grandezze di interesse in un punto $\overline{x}$ fissato
			
			\item Lagrangiano: seguo una particella fluida (= un punto materiale) nella sua evoluzione, nel suo moto che quindi è in una $\x(t)$
		\end{itemize}
		Le grandezze che osservo sono quindi grandezze lagrangiane $F(\x(t),t)$ che dipende da t esplicitamente \underline{e} implicitamente attraverso $\x(t)$
		
		Data $F(\x,t)$ grandezza estensiva per unità di massa (udm). Per consolidare la sua derivata temporale seguendo la particella fluida che a t si trova in $\x(t)$ in moto, dobbiamo considerare che F dipende dal tempo sia esplicitamente che attraverso $\x(t)$, cioè $F=F(\x,t)$, quindi la derivata è la derivata di funzione composta a più variabili e si indica come 
		$$
		\dfrac{D}{Dt} =\quad \textrm{DERIVATA SOSTANZIALE o MATERIALE o CONVETTIVA}.
		$$
		Per la regola delle derivate di funzione composta
		$$
		\dfrac{D}{Dt} F(\xt,t) = \dfrac{\partial}{\partial t}F(\x,t) + \sum_{i=1}^{3} \underset{(grad F)_i}{\underbrace{\dfrac{\partial F(\x,t)}{\partial x_i}}} \underset{v_i(\x,t)}{\underbrace{\dfrac{dx_i(t)}{dt}}}
		$$
		\begin{equation}
			\dfrac{D}{Dt} F(\xt,t) = \dfrac{\partial F}{\partial t} + \bar{v}(\x,t) \cdot \bar{F}(\x,t)
		\end{equation}
		e questa gode delle proprietà di tutte le derivate ordinarie; per esempio la variazione DF della F seguendo il continuo nel suo moto naturale è
		\begin{equation}
			DF(\x,t) = F(\x(t+dt),t+dt) - F(\xt,t) = \dfrac{D}{Dt} F(\xt,t) dt
		\end{equation}
		al primo ordine (relazione tra differenziali e derivata sostanziale)\\
		Dim: $F(\xt,t) = F(g(x))$ con $g(t) = g(x_1(t),x_2(t),x_3(t),t)$
		$$
		\overset{\cdot}{g}(t) = \dfrac{d}{dt} g(t) = \left(\dfrac{dx_1}{dt},\dfrac{dx_2}{dt},\dfrac{dx_3}{dt},1\right) = (v_1,v_2,v_3,1)
		$$
		$$
		\implies \dfrac{d}{dt}F(g(t)) = \dfrac{\partial F}{\partial g_1}\overset{\cdot}{g}_1 + \dfrac{\partial F}{\partial g_2}\overset{\cdot}{g}_2 + \dfrac{\partial F}{\partial g_3}\overset{\cdot}{g}_3 + \dfrac{\partial F}{\partial g_4}\overset{\cdot}{g}_4 
		$$
		$$
		 = \dfrac{\partial F}{\partial x_1}v_1 + \dfrac{\partial F}{\partial x_2}v_2 + \dfrac{\partial F}{\partial x_3}v_3 + \dfrac{\partial F}{\partial t} = \dfrac{\partial F}{\partial t} + \vel \cdot grad \bar{F}(\x,t)
		$$
		Si tratta di una derivata fatta seguendo il moto della particella di continuo
		
	

\chapter{Derivata sostanziale di integrali}
	\section{Derivata sostanziale degli integrali di volume}
			Dato un continuo e presa una regione R(t) che al tempo t è occupata da parte di questo continuo, consideriamo una grandezza estensiva che prende il valore $\mathbb{F}(t)$ sul volume R(t); la grandezza per unità di volume associata a $\mathbb{F}(t)$ sia chiamata $F(\xt,t)$, ovvero vale
			\begin{equation}
				\mathbb{F}(t) = \int_{R(t)} F(\xt,t)d^3x
			\end{equation}
			$\mathbb{F}(t)$ dipende dal tempo sia esplicitamente, sia implicitamente perché seguendo il continuo nel suo moto naturale, R(t) evolve nel tempo la derivata temporale di $\mathbb{F}(t)$ è una derivata sostanziale. Si dimostra che
			\begin{equation}
			\dfrac{D}{Dt} \mathbb{F}(t) = \int_{R(t)} \left[ \dfrac{D}{Dt}F(\xt,t) + F(\xt,t) \cdot div\vel(\xt,t) \right]d^3 x = \int_{R(t)} \left[ \dfrac{\partial}{\partial t}F(\xt,t) + \mathrm{div}\left(F(\x,t) \cdot \vel(\x,t)\right) \right]d^3 x
			\end{equation}
			
			Dim: per definizione di derivata e di $\mathbb{F}(t)$
			$$
			\dfrac{D\mathbb{F}(t)}{Dt} = \lim_{\Delta t \to 0} \dfrac{1}{\Delta t} \left[\int_{R(t+\Delta t)} F(\x(t+\Delta t),t+\Delta t)d^3x - \int_{R(t)} F(\xt,t)d^3x \right]
			$$
			Detto $\xp \defeq \x(t+\Delta t)$, al primo ordine in t si può scrivere $\xp = \xp(\x) = \x + \vel \Delta t$.\\
			Rielaboriamo dunque il primo integrale fatto a $t + \Delta t$ su $R(t+\Delta t)$:
			$$
			\int_{R(t+\Delta t)} F(\x(t+\Delta t),t+\Delta t)d^3x = \int_{R'} F(\xp,t+\Delta t) d^3x' = \int_{R} F(\xp(\x),t+\Delta t) \left|\mathbf{J}(x'|x)\right|d^3x
			$$
			avendo usato il teorema del cambio di variabile per fare il cambio $\xp \rightarrow \x$, ed essendo $\mathbf{J}(x'|x)$ Jacobiano della trasformazione, che si può elaborare sviluppando al I ordine $\xp$.
			$$
			\left|\mathbf{J}(x'|x)\right| = det\left(\dfrac{\partial x_i'}{\partial x_j}\right) = det \left(\delta_{ij} + \Delta t \dfrac{\partial v_i(\x,t)}{\partial x_j}\right) = 1 + \Delta t Tr \left(\dfrac{\partial v_i(\x,t)}{\partial x_j}\right)
			$$
			sfruttando $det\left(\mathbb{I} + \phi \underline{\underline{A}}\right) = 1 + \phi Tr\left(\underline{\underline{A}}\right)$ se $\phi ( =\Delta t)$ infinitesimo. Esplicitando la traccia si ha:
			$$
			\rightarrow = 1+ \Delta t \dfrac{\partial v_i(\x,t)}{\partial x_j} = 1 + \Delta t div(\vel(\x,t))
			$$
			Per l'espressione già nota del differenziale $DF(\xt,t)$, si può scrivere al primo ordine in t
			$$
			F(\x(t+\Delta t),t+\Delta t) = F(\xt,t) + \dfrac{D}{Dt}F(\xt,t) \Delta t
			$$
			e in definitiva riscriviamo la derivata totale di $\mathbb{F}(t)$
			
			$$
			\dfrac{D\mathbb{F}(t)}{Dt} = \lim_{\Delta t \to 0} \dfrac{1}{\Delta t} \left[\int_{R(t+\Delta t)} F(\x(t+\Delta t),t+\Delta t)d^3x - \int_{R(t)} F(\xt,t)d^3x \right]=
			$$
			$$
			= \lim_{\Delta t \to 0} \dfrac{1}{\Delta t} \left\{\int_{R(t)} \left[F(\xt,t) + \dfrac{D}{Dt}F(\xt,t) \Delta t\right]\left[1 + \Delta t \cdot \mathrm{div}(\vel(\xt,t)) \right] d^3x - \int_{R(t)} F(\xt,t) d^3x \right\}=
			$$
			
			$$
			= \lim_{\Delta t \to 0} \dfrac{1}{\Delta t} \left\{\int_{R(t)} \left[F(\xt,t)\mathrm{ div}\left(\vel(\xt,t)\right) \Delta t + \dfrac{DF(\xt,t)}{Dt} \Delta t + \dfrac{F(\xt,t)}{Dt}\mathrm{ div}\left(\vel(\xt,t)\right)(\Delta t)^2  \right] d^3x\right\} =
			$$
			
			$$
			= \int_{R(t)} \left[\dfrac{DF(\xt,t)}{Dt} + F(\xt,t) \mathrm{ div}\left[\vel(\xt,t)\right]\right]d^3x
			$$
			trascurando i termini al 2 ordine in $\Delta t$ e risolvendo il limite.\\
			Esplicitando la derivata totale di F:
			
			$$
			= \int_{R(t)} \left[\dfrac{\partial F(\xt,t)}{\partial t} + \underset{\mathrm{ div}\left(F\vel\right)}{ \underbrace{\left(\vel\cdot \mathrm{grad}\right)F(\x,t)  + F(\x,t)\mathrm{ div}\vel}}\right]d^3x
			$$
			\begin{equation}
				= \int_{R(t)} \dfrac{\partial F(\xt,t)}{\partial t} + \mathrm{ div}\left(F(\xt,t)\vel(\xt,t)\right)d^3x
			\end{equation}
			L'espressione ha validità generale per ogni grandezza estensiva F. Il primo e fondamentale risultato è l'applicazione $F=\rho$ (densità) e la conseguente equazione di continuità, o conservazione della massa.
		
	
		\section{Equazione di continuità - condizione analitica di incomprimibilità}
		Presa una porzione di continuo racchiusa in una regione $R(t)$ al tempo t, seguendola nel suo moto naturale sappiamo che la massa M verrà conservata. Dato il legame con la densità $\rho(\x,t)$ diciamo quindi
		\begin{equation}
			M(t) = \int_{R(t)} \rho(\xt,t)d^3x = \mathrm{costante}
		\end{equation}
		ovvero $\frac{dM(t)}{dt} = 0 $ seguendo l'evoluzione naturale della porzione nella $R(t)$, che vuol dire
		$$
		\int_{R(t)} \rho(\xt,t)d^3x = \int_{R(t)}\left[ \dfrac{\partial \rho(\xt,t)}{\partial t} + \mathrm{ div}\left(\rho(\xt,t) \vel(\xt,t)\right)\right]d^3x = 0
		$$
		Data l'arbitrarietà con cui possiamo prendere $R(t)$, l'integranda deve essere nulla:
		
		\begin{equation}
			 \dfrac{\partial \rho(\xt,t)}{\partial t} + \mathrm{ div}\left(\rho \vel\right) = 0
		\end{equation}
		che, come visto, si può anche scrivere
		$$
		\dfrac{D\rho}{Dt} + \rho\mathrm{ div}(\vel) \quad oppure \quad \dfrac{1}{\rho}\dfrac{D\rho}{Dt} = -\mathrm{ div}(\vel) \quad oppure \quad \dfrac{1}{v}\dfrac{Dv}{Dt} = \mathrm{ div}(v)
		$$
		con $v\defeq\frac{1}{\rho}$ volume specifico (volume per unità di massa).\\
		Da questa si ricava la \underline{condizione di incomprimibilità} ( $\rho$ costante): $\mathrm{ div}(\vel) = 0$\\
		
		Nota: sempre sfruttando la legge per la derivazione degli integrali di volume, se $F(\x,t)=1$, si ha $\mathbb{F}(t) = \int_R 1\cdot d^3x = V$ volume della regione R, e
		$$
		\dfrac{dV}{dt} = \int_{R(t)}\left[\dfrac{\partial}{\partial t}(1) + \mathrm{ div}(1\cdot \vel)\right] d^3x = \int_{R(t)}\mathrm{div}(\vel) d^3x = \int_{\partial R(t)} \vel\cdot \hat{n} da
		$$
		dove $\partial R(t)$ è la superficie che racchiude $R(t)$, con $\hat{n}$ versore normale uscente.
		Questo ci dice che la variazione di volume di un continuo nel tempo è data dal flusso di velocità attraverso il contorno del volume stesso.

	
	\section{Leggi in forma integrale e locale}
		Le leggi della fisica vengono espresse in forma integrale oppure in forma locale. \eacc importante chiarire il legame tra queste due forme, che risultano, come ovvio, equivalenti.
		
		Una legge in forma \underline{integrale}, per quanto concerne l'approccio seguito qui, prevede l'uguaglianza tra la derivata sostanziale di un integrale, e dall'altra parte un altro integrale.
		
		Esempio: scriviamo la prima equazione cardinale della dinamica per una porzione di continuo di densità $\rho(\x,t)$ contenuta nella regione $R(t)$, sottoposta a forze la cui risultante è $F(t)$. 
		L'equazione è:
		
		$$
		\dfrac{D}{Dt}\int_{R(t)}\rho(\x,t)\vel(\x,t)d^3x = \bar{F}
		$$
		
		A sua volta possiamo esprimere $\bar{F}$ usando $\bar{f} = $ forza per udM e integrando $\rho \bar{f}$, che è una forza per unità di volume (udV), sul volume $R(t)$:
		
		$$
		\bar{F}(t) = \int_{R(t)} \rho(\x,t)\bar{f}(\x,t)d^3x 
		$$
		da cui
		
		$$
		\dfrac{D}{Dt}\int_{R(t)}\rho(\x,t)\vel(\x,t)d^3x = \int_{R(t)} \rho(\x,t)\bar{f}(\x,t)d^3x
		$$
		
		Il primo integrale, poiché $\rho$ soddisfa l'equazione di continuità, risulta scrivibile in altra maniera (la vediamo appena più sotto in generale):
		
		$$
		\dfrac{D}{Dt}\int_{R(t)}\rho(\x,t)\vel(\x,t)d^3x =
		\int_{R(t)}\rho \dfrac{D\vel}{Dt}d^3x
		= \int_{R(t)} \rho(\x,t)\bar{f}(\x,t)d^3x
		$$
		
		e dato che $R(t)$ è una regione arbitraria, devono essere uguali le funzioni integrande (e per $\rho \neq 0$ la si può semplificare) 
		
		$$
		\dfrac{D\vel}{Dt}= \bar{f}
		$$
		che esprime la stessa legge integrale in forma locale, tra grandezze per udM ($\bar{f}=$ F per udM, $\vel=$ quantità di moto per udM).
		
		Come accennato, il risultato è generalizzabile. Prendiamo infatti una grandezza estensiva g, grandezza per udM
		
		$$
		\dfrac{D}{Dt}\int_{R}\rho g d^3x = \int_{R} \left[\dfrac{\partial (\rho g)}{\partial t} + \mathrm{ div}(\rho g \vel)\right]d^3x=
		$$
		$$
		=\int_{R}\left[\rho\dfrac{\partial g}{\partial t} + g  \dfrac{\partial \rho}{\partial t} + g\mathrm{ div}(\rho\vel) + \rho \vel \cdot \mathrm{grad} g\right] d^3x=
		$$
		$$
		\int_{R} \left\{g\underset{=0 \textrm{ eq continuità}}{\underbrace{\left[\dfrac{\partial \rho}{\partial t} + \mathrm{ div}(\rho\vel)\right]}} +
		\rho\underset{\defeq =  Dg/Dt}{ \underbrace{\left[\dfrac{\partial g}{\partial t} + \left(\vel \cdot \mathrm{grad}\right)g\right]}} \right\}d^3x
		$$
		\begin{equation}
			\dfrac{D}{Dt}\int_{R}\rho g d^3x = \int_{R} \rho \dfrac{Dg}{D t}
		\end{equation}
		
		Quindi se esiste h grandezza per udM tale che esiste una legge integrale
		
		$$
		\dfrac{D}{Dt}\int_{R} \rho g d^3x = \int_{R} \rho h \quad \implies \quad \int_{R}\rho \dfrac{Dg}{Dt}d^3x = \int_{R} \rho h d^3x
		$$
		
		questa è equivalente alla legge locale
		
		\begin{equation}
			\dfrac{D}{Dt}g = h
		\end{equation}
		
		Nota: si osservi che troviamo un legame tra la derivata parziale di una grandezza per udV, $\rho g$, e la derivata sostanziale della stessa grandezza per udM, g,
		
		\begin{equation}
			\dfrac{\partial(\rho g)}{\partial t} + \mathrm{ div}\left(\rho g \vel\right) = \rho \dfrac{Dg}{Dt}
		\end{equation}
		
	\section{Derivata sostanziale degli integrali di linea}
		
		Si dimostra che data una curva $\gamma(t)$m, che rappresenta un insieme di punti del continuo in moto naturale, e quindi curva che dipende dal tempo, presa la grandezza $f(\x,t)$
		
		\begin{equation}
			\dfrac{D}{Dt}\int_{\gamma(t)} f(\x,t) ds_i = \int_{\gamma(t)} \left(f\dfrac{\partial v_i}{\partial x_j} + \dfrac{D}{Dt}f\delta_{ij}\right)ds_j
		\end{equation}
		
		dove $ds_i$ è la i-esima componente dello spostamento $d\x$ lungo la curva e $v_i$ la componente i-esima della velocità $\vel$.
		Dimostrazione: $\gamma$ è una funzione a valori in $\mathbb{R}^3$ definita su un intervallo $\left[a,b\right]$ della retta reale; la si può parametrizzare in vari modi equivalenti con un parametro corrente $\alpha \in \left[a,b\right]$ che in a e b dà gli estremi della curva.
		
		Scegliamo quindi una parametrizzazione, che si esprime con $\x=\x(\alpha)$, di $\alpha \in \left[a,b\right]$, funzione regolare di componenti i-esime $x_i(\alpha)$; l'integrale di linea è definito dall'integrale di Riemann ordinario su $\left[a,b\right]$
		
		$$
		\int_{\gamma} f(\x)ds_i = \int_{a}^{b} f\left(\x(\alpha)\right)\dfrac{dx_i(\alpha)}{d\alpha}d\alpha
		$$
		
		la cui derivata sostanziale è
		$$
		\dfrac{D}{Dt}\int_{\gamma(t)} f(\x,t)ds_i =
		\lim_{\Delta t\to 0} \dfrac{1}{\Delta t}\left[\int_{\gamma(t+\Delta t)} f(\x,t+\Delta t)ds_i + \int_{\gamma(t)}f(\x,t)ds_i\right]
		$$
		(si noti che $\gamma$ varia essa stessa).
		
		Scegliamo una parametrizzazione opportuna: al tempo t $\x=\x(\alpha,t)$, e al tempo $t+\Delta t$ in modo che al valore di $\alpha$ corrisponda $\x(\alpha,t+\Delta t)=$ evoluto temporale del punto $\x$ della curva al tempo t, $\x(\alpha,t)$ dato dallo stesso $\alpha$.\\
		In questo modo, vale al primo ordine in $\Delta t$
		$$
		\x(\alpha,t+\Delta t) = \x(\alpha,t) + \vel(\x(\alpha,t))\Delta t
		$$
		e rielaboriamo la differenza tra integrali
		
		$$
		\int_{\gamma(t+\Delta t)}f(\x,t+\Delta t)ds_i - \int_{\gamma(t)}f(\x,t)ds_i
		$$
		
		applicando la parametrizzazione
		
		$$
		\int_{a}^{b} f(\x(\alpha,t+\Delta t), t+\Delta t)\dfrac{dx_i(\alpha,t+\Delta t)}{d\alpha}d\alpha - \int_{a}^{b} f(\x(\alpha,t),t)\dfrac{dx_i(\alpha,t)}{d\alpha}d\alpha
		$$
		
		$$
		=\int_{a}^{b} f(\x(\alpha,t+\Delta t), t+\Delta t)\dfrac{\partial x_i(\alpha,t+\Delta t)}{\partial x_j(\alpha,t)}\dfrac{dx_j(\alpha,t)}{d\alpha}d\alpha - \int_{a}^{b} f(\x(\alpha,t),t)\dfrac{dx_j(\alpha,t)}{d\alpha}\delta_{ij}d\alpha=
		$$
		
		$$
		=\int_{a}^{b}\left[f(\x(\alpha,t+\Delta t), t+\Delta t)\dfrac{\partial x_i(\alpha,t+\Delta t)}{\partial x_j(\alpha,t)} - f(\x(\alpha,t),t)\delta{ij}\right]\dfrac{dx_j(\alpha,t)}{d\alpha}d\alpha =
		$$
		
	
	
	
\chapter{Sforzi; fluidostatica: isotropia e continuità della pressione}
	\section{Tensore degli sforzi}			
		
	\section{Sforzi (stresses)}
		\subsection{Sforzo normale e sforzo tangenziale}			
	
	\section{Fluidostatica}
		\subsection{Isotropia della pressione in equilibrio}			
		\subsection{Continuità di p all'interfaccia}
		\subsection{Tensione superficiale}
	

\chapter{Equazione di Eulero}
	\section{Appendice: teorema della divergenza per casi particolari}
		
	

\chapter{Fluidostatica: equilibrio meccanico, equilibrio e stabilità dell'atmosfera}
	\section{Ritorno alla fluidostatica}
	\section{Equilibrio dell'atmosfera - stabilità dell'equilibrio}
		\subsection{Equilibrio meccanico + termico (atmosfera isoterma secca)}		
		\subsection{Atmosfera isoentropica (secca)}
					

\chapter{Fluidostatica di fluidi incomprimibili}
	\section{Barometro di Torricelli}
	\section{Vasi comunicanti - fluidi immiscibili}
	\section{Pressa idraulica}
	\section{Forza di Archimede}
	\section{Centro di spinta ed equilibrio}
	\section{Isotropia}
			
		
\chapter{Fluidodinamica di fluidi perfetti; flusso di impulso e di energia}
	\section{Linee di corrente (Stream lines)}
	\section{Flusso di quantità di moto}
	\section{Forza su un tubo a gomito}
	\section{Flusso di energia}
		\subsection{Flusso di energia in campo esterno}
			
\chapter{Flusso stazionario: equazione di bernoulli e applicazioni}
	\section{Equazione di Bernoulli}
		\subsection{Fluido incomprimibile}
		\subsection{Fluido reale}	
	\section{Teorema di Torricelli}
		\subsection{Tubo di Venturi}
		\subsection{Cavitazione}
		\subsection{Tubo di Pitot - sistema Pitot-statico}
		\subsection{Sifone}
		\subsection{Portanza}
		\subsection{Volo aereo}
				
\chapter{Teorema di Kelvin; flusso potenziale}
	\section{Teorema di Kelvin - conservazione della circolazione}
	
	Si era visto che data la curva $\gamma(t)$, vale
	
	$$
		\dfrac{D}{Dt}\int_{\gamma(t)} f(\x,t)ds_i = \int_{\gamma(t)} f(\x,t)\left(\dfrac{\partial v_i}{\partial x_j} + \dfrac{D}{Dt}f(\x,t)\delta_{ij}\right)ds_j
	$$
	e se $f=f_i$ i-esima componente di $\bar{f}(\x,t)$ vettoriale
	
	$$
	\dfrac{D}{Dt} \int_{\gamma(t)}\bar{f}\cdot d\x = \int_{\gamma(t)}\bar{f}\cdot \vel + \int_{\gamma(t)}\dfrac{D}{Dt}\bar{f}\cdot d\x
	$$
	
	dunque se $\bar{f}=\vel$ velocità di un fluido perfetto, e $\gamma(t)$ linea ideale costituita da punti del fluido nel loro moto
	
	$$
	\dfrac{D}{Dt}\int_{\gamma(t)}\vel\cdot d\x = 
	\int_{\gamma(t)}\vel\cdot d\vel + \int_{\gamma(t)}\dfrac{D\vel}{Dt}\cdot d\x = 
	\int_{\gamma(t)} d\left(\dfrac{1}{2}v^2\right) + \int_{\gamma(t)} \dfrac{D\vel}{Dt}\cdot d\x
	$$
	
	Se $\gamma(t)$ è una curva chiusa, si definisce $\Gamma$ circuitazione o circolazione $\Gamma \defeq \oint_{\gamma(t)}\vel \cdot d\bar{l}$ e il primo integrale, essendo l'integrale di differenziale esatto, è nullo
	
	$$
	\dfrac{D \Gamma}{Dt} = \dfrac{D}{Dt}\oint_{\gamma(t)} \vel\cdot d\bar{l} = \oint_{\gamma(t)}\dfrac{D\vel}{Dt} \cdot d\bar{l}
	$$
	
	per Eulero $\dfrac{D\vel}{Dt} = -\dfrac{1}{\rho} \mathrm{grad}p - \mathrm{grad}u$.
	Nell'ipotesi di \underline{moto isoentropico}, oppure nel caso in cui  $ -\dfrac{1}{\rho}\mathrm{grad}p$ ammetta potenziale:
	
	$$
	\dfrac{D\vel}{Dt} = -\mathrm{grad}(w+u)
	$$
	
	$$
	\implies \dfrac{D\Gamma}{Dt} = \oint_{\gamma(t)}\dfrac{D\vel}{Dt} \cdot d\bar{l} = 
	-\oint_{\gamma(t)}\mathrm{grad}(w+u) \cdot d\bar{l} = 
	-\int_{S}\textrm{ rot}\left[\mathrm{grad}(w+u)\right]\cdot d\bar{s}
	$$
	usando il teorema di Stokes nell'ultimo passaggio, ed essendo S superficie sottesa da $\gamma$.
	Poiché rot(grad(f))=0 per ogni f, la circolazione $\Gamma$ si conserva nel moto isoentropico di fluido perfetto (Teorema di Kelvin)
	\begin{equation}
		\dfrac{D\Gamma}{Dt} = \oint_{\gamma(t)}\vel\cdot d\bar{l} = 0
	\end{equation}
	
	prendendo una $\gamma$ che sottenda un'area infinitesima, ci restringiamo praticamente all'intorno di una singola linea di corrente; dunque la \underline{vorticità} (rot$\vel$) si conserva nel flusso, si muove col moto naturale del fluido
	
	$$
	\oint_{\gamma = \partial S} \vel \cdot d\bar{l} = \int_{S}\mathrm{rot}\vel\cdot d\bar{a} \approxeq \mathrm{rot}\vel \cdot \bar{S} = costante
	$$
	
	Nota: come Landau ci mettiamo nell'usuale caso isoentropico; basta in verità che $\dfrac{1}{\rho}\mathrm{grad}p$ ammetta potenziale $\psi$, così che $\dfrac{1}{\rho}\mathrm{grad}p = \mathrm{grad}\psi$; sono diversi i casi possibili:
	\begin{itemize}
		\item isoentropico: $dw = Tds + \dfrac{1}{\rho}p = \dfrac{1}{\rho}p \implies \psi = w$ entalpia
		\item isotermo: $dp = -sdT + \dfrac{1}{\rho}p = \dfrac{1}{\rho}p \implies \psi=\phi$ energia libera di Gibbs
		\item incomprimibile: $\dfrac{1}{\rho}p = \mathrm{grad}\left(\dfrac{p}{\rho}\right) \implies \psi = \dfrac{p}{\rho}$
	\end{itemize}
	
	\section{Flusso potenziale}
	La conservazione della circolazione $\Gamma = \oint_{\gamma(t)} \vel \cdot d\bar{l}$ porta a osservazioni significative.
	Nel caso di flusso stazionario, se vi è un punto per  il quale $\mathrm{rot}\vel = 0$, presa la linea di corrente che vi passa e considerata una $\gamma$ infinitesima che la racchiude si può affermare che $\mathrm{rot}\vel =0$ su tutta la linea.
	Se il moto non è stazionario, il risultato vale lungo una traiettoria; ciò sempre tenendo a mente l'ipotesi di flusso isoentropico, altrimenti l'osservazione non è valida ( per es. in caso di urti o turbolenza).
	
	Se supponiamo di avere un fluido in moto isoentropico che all'infinito è uniforme ($\vel$ uniforme; caso particolare fluido inizialmente a riposo, con $\vel$ nulla e $\implies$ uniforme; è potenziale e tale resta), $\mathrm{rot}\vel =0$ \underline{ovunque}.
	
	Il fatto che $\mathrm{rot}\vel =0$ dice che è esprimibile come gradiente di un potenziale scalare $\varphi : \vel = \mathrm{grad}\varphi$.
	$\varphi$ è il POTENZIALE SCALARE DI VELOCIT$\grave{A}$ e il moto è nullo.
	
	\subsection{Flusso potenziale o irrotazionale}
	
	In questo flusso non esistono linee di corrente chiuse
	$$
	\oint_{\gamma} \vel \cdot d\bar{l} = 0
	$$
	Avendo $\vel = \mathrm{grad}\varphi $ e considerando $- 1\over \rho \mathrm{grad}p = - \mathrm{grad}\psi$, l'equazione di Eulero
	
	$$
	\dfrac{\partial \vel}{\partial t} + \left(\vel \cdot \mathrm{grad}\right)\vel = -\mathrm{grad}\left(\psi + u\right)
	$$
	
		\subsection{Flusso attorno ad un ostacolo}
				
	\section{Condizione di incomprimibilità - una prospettiva fisica}
		\subsection{Flusso stazionario}				
 		\subsection{Flusso non stazionario}		
	\section{Forza di resistenza nel flusso potenziale oltre a un corpo}		


\chapter{Onde di gravità}
	\section{Condizioni cinematiche generali}
		\subsection{Fluidi perfetti}
	\section{Condizioni dinamiche generali}
		\subsection{Fluidi perfetti}
		\subsection{Flusso potenziale}
	\section{Linearizzazione delle condizioni all'interfaccia}
	\section{Onde di gravità in un bacino di profondità infinita}
	\section{Onde di gravità in un bacino di profondità finita}
	\section{Onde di gravità tra due fluidi limitati in altezza}


\chapter{Trasporto di energia in onde di gravità; appendici}
	\section{Appendice - velocità di fase e di gruppo (momento minimo)}
	\section{Appendice - Vademecum minimo di funzioni iperboliche}

\chapter{Fluidi reali: tensori dei gradienti delle velocità e degli sforzi, equazione di Navier-Stokes}

	\section{Tensore dei gradienti delle velocità - decomposizione e significato geometrico}
	\section{Fluidi reali (viscosi) - relazione di Cauchy}
	\section{Tensore degli sforzi Newtoniano - equazione di Navier-Stokes}


\chapter{Equazione di Navier-Stokes in coordinate non cartesiane}

	\section{Coordinate cilindriche}
	\section{Coordinate sferiche}


\chapter{Momento angolare e considerazioni sulla simmetria del tensore degli sforzi}

	\section{Equazioni cardinali della dinamica - momento angolare}
	\section{Momento delle forze esterne}
	\section{Momento angolare orbitale e di spin}
	
	\section{Pressione in un fluido in moto - pressione meccanica, termodinamica ed equilibrio locale}
	\section{Flusso di quantità di moto - tensore densità di flusso di qdm}
	\section{Dissipazione nel fluido viscoso}
	
\chapter{Esempi di flusso viscoso}

	\section{Flusso di Poiseuille}
		\subsection{Geometria piana}
		\subsection{Geometria cilindica - flusso in conduttura}
	\section{Flusso di Couette}
		\subsection{Geometria piana}
		\subsection{Geometria cilindica}
	
\chapter{Leggi di similarità e numeri adimensionati}

	\section{Similitudine geometrica}
	\section{Similitudine cinematica}
	\section{Problemi simili}
		\subsection{Problemi dinamicamente simili}
		\subsection{Numero di Froude}
		\subsection{Numero di Strouhal}
	
\chapter{Problema di Stokes per il moto di una sfera in un fluido viscoso}

	\section{Confronto con sfera in fluido perfetto (moto potenziale)}
	\section{Campo di pressione}
	\section{Forza applicata sulla sfera}
	\section{Perfezionamento della formula di Stokes}
		\subsection{Formula di Oseen e coefficiene di penetrazione}
		
\chapter{Moti oscillatori in fluidi viscosi}

	\section{Fluido di profondità infinita su piano oscillante}
	\section{Fluido limitato tra due piani}
	\section{Strato di fluido con pelo libero}
	\section{Corpo generico oscillante in fluido viscoso}

\chapter{Smorzamento di onde di gravità; correnti superficiali: Ekman layer}

	\section{Ekman layer}
	
	
\end{document}